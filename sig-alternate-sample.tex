\documentclass{sig-alternate-05-2015}

\begin{document}

% Copyright
%\setcopyright{acmcopyright}
%\setcopyright{acmlicensed}
%\setcopyright{rightsretained}
%\setcopyright{usgov}
%\setcopyright{usgovmixed}
%\setcopyright{cagov}
%\setcopyright{cagovmixed}

\title{Towards an Automatic Conversion of SPARQL queries to Gremlin Traversals}
%\title{Needs a better, witty title}


\numberofauthors{3} %  in this sample file, there are a *total*
% of EIGHT authors. SIX appear on the 'first-page' (for formatting
% reasons) and the remaining two appear in the \additionalauthors section.
%

\author{
% 1st. author
\alignauthor
Harsh Thakkar\\
       \affaddr{Enterprise Information Systems Lab}\\
       \affaddr{University of Bonn}\\
       \affaddr{Bonn, Germany}\\
       \email{hthakkar@uni-bonn.de}
% 2nd. author
\alignauthor
Second Author\\
       \affaddr{Enterprise Information Systems Lab}\\
       \affaddr{University of Bonn}\\
       \affaddr{Bonn, Germany}\\
       \email{email@uni-bonn.de}
% nth. author
\alignauthor N$^_{th}$ Author\\
       \affaddr{Name of Lab or Group}\\
       \affaddr{Name of University}\\
       \affaddr{City, Country}\\
       \email{name@email.com}
}

%\additionalauthors{Additional authors: John Smith (The Th{\o}rv{\"a}ld Group,
%email: {\texttt{jsmith@affiliation.org}}) and Julius P.~Kumquat
%(The Kumquat Consortium, email: {\texttt{jpkumquat@consortium.net}}).}

%\date{30 July 1999}


\maketitle
\begin{abstract}
The abstract goes here \dots
\end{abstract}


%
% The code below should be generated by the tool at
% http://dl.acm.org/ccs.cfm
% Please copy and paste the code instead of the example below. 
%
%\begin{CCSXML}
%\end{CCSXML}



%
% End generated code
%

%
%  Use this command to print the description
%
\printccsdesc

% We no longer use \terms command
%\terms{Theory}

\keywords{ACM proceedings; \LaTeX; text tagging}

%--------------INTRODUCTION----------------
\section{Introduction}
\subsection{Motivation}
\subsection{Research Questions}

%--------------QUERY LANGUAGE ANALYSIS----------------
\section{Query Language Hueristics}
\subsection{SPARQL}
\subsubsection{Underlying model and complexity}
\subsection{Gremlin}
\subsubsection{Underlying model and complexity}

\section{SPARQL-Gremlin Mapper}


%--------------DATA DESCRIPTION----------------
\section{Data}
\subsection{DBpedia dataset}
\subsubsection{Data models}
\begin{enumerate}
    \item RDF data model
    \item Graph data model
\end{enumerate}

\subsection{Northwind dataset}
\subsubsection{Data models}
\begin{enumerate}
    \item RDF data model
    \item Graph data model
\end{enumerate}


%--------------TOOLS----------------
\section{Tools}
%----------CAN BE MERGED/MOVED INTO A NEW SECTION: EXPERIMANTAL SETUP ------------
\subsection{Openlink Virtuoso}
\subsection{Neo4J}



%-------------EXPERIMENTS---------------
\section{Experiments}
\subsection{Setup}
\subsection{Queries}
\subsubsection{QALD 6 queries}
\subsubsection{Northwind queries}
\subsection{Evaluation}
We use QALD 6 and the Northwind queryset in order to evaluate the performance of our approach.

\section{Related work}

\section{Conclusion}

%ACKNOWLEDGMENTS are optional
\section{Acknowledgments}

\bibliographystyle{abbrv}
\bibliography{sigproc}  % sigproc.bib is the name of the Bibliography in this case

%APPENDICES are optional
%\balancecolumns
\appendix
%Appendix A

\section{Headings in Appendices}

\end{document}